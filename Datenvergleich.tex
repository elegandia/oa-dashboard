% Options for packages loaded elsewhere
\PassOptionsToPackage{unicode}{hyperref}
\PassOptionsToPackage{hyphens}{url}
%
\documentclass[
]{article}
\usepackage{amsmath,amssymb}
\usepackage{lmodern}
\usepackage{ifxetex,ifluatex}
\ifnum 0\ifxetex 1\fi\ifluatex 1\fi=0 % if pdftex
  \usepackage[T1]{fontenc}
  \usepackage[utf8]{inputenc}
  \usepackage{textcomp} % provide euro and other symbols
\else % if luatex or xetex
  \usepackage{unicode-math}
  \defaultfontfeatures{Scale=MatchLowercase}
  \defaultfontfeatures[\rmfamily]{Ligatures=TeX,Scale=1}
\fi
% Use upquote if available, for straight quotes in verbatim environments
\IfFileExists{upquote.sty}{\usepackage{upquote}}{}
\IfFileExists{microtype.sty}{% use microtype if available
  \usepackage[]{microtype}
  \UseMicrotypeSet[protrusion]{basicmath} % disable protrusion for tt fonts
}{}
\makeatletter
\@ifundefined{KOMAClassName}{% if non-KOMA class
  \IfFileExists{parskip.sty}{%
    \usepackage{parskip}
  }{% else
    \setlength{\parindent}{0pt}
    \setlength{\parskip}{6pt plus 2pt minus 1pt}}
}{% if KOMA class
  \KOMAoptions{parskip=half}}
\makeatother
\usepackage{xcolor}
\IfFileExists{xurl.sty}{\usepackage{xurl}}{} % add URL line breaks if available
\IfFileExists{bookmark.sty}{\usepackage{bookmark}}{\usepackage{hyperref}}
\hypersetup{
  pdftitle={Datenvergleich},
  hidelinks,
  pdfcreator={LaTeX via pandoc}}
\urlstyle{same} % disable monospaced font for URLs
\usepackage[margin=1in]{geometry}
\usepackage{graphicx}
\makeatletter
\def\maxwidth{\ifdim\Gin@nat@width>\linewidth\linewidth\else\Gin@nat@width\fi}
\def\maxheight{\ifdim\Gin@nat@height>\textheight\textheight\else\Gin@nat@height\fi}
\makeatother
% Scale images if necessary, so that they will not overflow the page
% margins by default, and it is still possible to overwrite the defaults
% using explicit options in \includegraphics[width, height, ...]{}
\setkeys{Gin}{width=\maxwidth,height=\maxheight,keepaspectratio}
% Set default figure placement to htbp
\makeatletter
\def\fps@figure{htbp}
\makeatother
\setlength{\emergencystretch}{3em} % prevent overfull lines
\providecommand{\tightlist}{%
  \setlength{\itemsep}{0pt}\setlength{\parskip}{0pt}}
\setcounter{secnumdepth}{-\maxdimen} % remove section numbering
\usepackage{booktabs}
\usepackage{longtable}
\usepackage{array}
\usepackage{multirow}
\usepackage{wrapfig}
\usepackage{float}
\usepackage{colortbl}
\usepackage{pdflscape}
\usepackage{tabu}
\usepackage{threeparttable}
\usepackage{threeparttablex}
\usepackage[normalem]{ulem}
\usepackage{makecell}
\usepackage{xcolor}
\ifluatex
  \usepackage{selnolig}  % disable illegal ligatures
\fi

\title{Datenvergleich}
\author{}
\date{\vspace{-2.5em}15.06.2021}

\begin{document}
\maketitle

\hypertarget{anzahl-publikationen}{%
\subsection{Anzahl Publikationen}\label{anzahl-publikationen}}

Der Datensatz der Medizinischen Bibliothek enthält \textbf{15178
Publikationen} für den Zeitraum 2018 bis 2020.

Der BIH-Datensatz enthält \textbf{13252 Publikationen} für den Zeitraum
2018 bis 2020.

\hypertarget{publikationen-in-den-datensuxe4tzen-nach-jahren}{%
\subsubsection{Publikationen in den Datensätzen nach
Jahren}\label{publikationen-in-den-datensuxe4tzen-nach-jahren}}

Auf die einzelnen Jahre des Untersuchungszeitraums bezogen ergibt sich
folgende Verteilung:

\begin{table}
\centering
\begin{tabular}{r|r|r}
\hline
year & medbib & bih\\
\hline
2018 & 4421 & 3906\\
\hline
2019 & 4676 & 4329\\
\hline
2020 & 6081 & 5017\\
\hline
\end{tabular}
\end{table}

\hypertarget{uxfcberschneidungen-und-abweichungen}{%
\subsection{Überschneidungen und
Abweichungen}\label{uxfcberschneidungen-und-abweichungen}}

\hypertarget{vergleich-der-datensuxe4tze-anhand-der-doi}{%
\subsubsection{Vergleich der Datensätze anhand der
DOI}\label{vergleich-der-datensuxe4tze-anhand-der-doi}}

Um die Überschneidungen und Abweichungen mit einem eindeutigen
Identifier analysieren zu können, wurden nur die Publikationen mit einer
DOI verwendet.

Keine DOI hatten 179 Publikationen der Medbib-Daten und 99 Publikationen
der BIH-Daten. Diese Publikationen wurden aus den Datensätzen entfernt.
Somit wurden 14999 Publikationen aus dem Medbib-Datensatz und 13153
Publikationen aus dem BIH-Datensatz ausgewertet.

Beide Datensätze wurden anhand der DOI miteinander verbunden. Der
verbundendene Datensatz enthält 15694 Publikationen. 12460 der
Publikationen sind in beiden Datensätzen enthalten.

Der Medbib-Datensatz enthält 2541 Publikationen, die nicht im
BIH-Datensatz enthalten sind.

Der BIH-Datensatz enthält wiederum 693 Publikationen, die nicht im
Medbib-Datensatz enthalten sind.

Die BIH-Daten, die nicht im Medbib-Datensatz enthalten sind, wurden in
der Datentabelle in der Spalte \texttt{in\_bih\_dataset} mit logischen
Operatoren gekennzeichnet.

\hypertarget{uxfcberpruxfcfung-des-oa-status}{%
\subsubsection{Überprüfung des
OA-Status}\label{uxfcberpruxfcfung-des-oa-status}}

In einem nächsten Schritt wurde der OA-Status überprüft.

981 von den 12460 in beiden Datensätzen enthaltenen Publikationen wurden
nicht mit dem identischen Open-Access-Status identifiziert.

Diese Datensätze wurden zur weiteren Überprüfung in der Datentabelle in
der Spalte \texttt{oa\_status\_identical\_medbib\_bih} mit logischen
Operatoren gekennzeichnet.

\hypertarget{datenvergleich-medbib-und-unpaywall}{%
\subsection{Datenvergleich Medbib und
Unpaywall}\label{datenvergleich-medbib-und-unpaywall}}

Der Vergleich der Medbib-Daten mit den Unpaywall-Daten ergab, dass der
der Open Access-Status in 282 Fällen nicht identisch ist.

Die Erscheinungsjahre sind in 1905 Fällen nicht identisch.

Die Abweichungen wurden in der Datentabelle in jeweils einer neuen
Spalte (\texttt{oa\_status\_identical\_medbib\_unpaywall} und
\texttt{year\_medbib\_unpaywall\_identical} mit logischen Operatoren
gekennzeichnet.

\end{document}
